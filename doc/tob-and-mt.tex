%   $Id: tob-and-mt.tex,v 1.1.1.1 2002/07/28 19:03:52 svanegmond Exp $    
\documentclass[12pt,a4paper]{article}

\begin{document}
\pagestyle{empty}

\parskip6pt
\parindent0pt

\def\cmd#1{\textsf{#1}}

\null
\vskip-3cm
\enlargethispage{3cm}
The following is a short description of a way to use \cmd{tob} combined
with \cmd{mt}.

Set the name of the tape in the \cmd{TAPE} environment variable like this
\begin{verbatim}
export TAPE=/dev/nftape
\end{verbatim}
such that we don't have to mention the name of the tape in each
command. The \cmd{export} command can be put in your
\verb+/etc/rc.d/rc.local+ file.

Before you use the tape at all it should be formatted. Go to DOS and
do that or buy preformatted tapes.

Retension the tape before you use it---this is strongly recommended by
the manufacturer of my tapes:
\begin{verbatim}
mt retension
\end{verbatim}

To make a full backup in the beginning of the tape be sure that the
position of the tape is at the beginning of the tape
\begin{verbatim}
mt rewind
tob -full <volume>
\end{verbatim}

If you later want to add to the tape set the position to the end of
your valid data before you execute the \cmd{tob} command
\begin{verbatim}
mt eod
tob -diff <volume>
\end{verbatim}

I you want to list the contents of the third backup made on the tape
position the tape at the beginning of the file after the second
backup; i.e.\ the third backup
\begin{verbatim}
mt rewind
mt fsf 2
tob -verbose 
\end{verbatim}

If you want to list the contents of the last backup on the tape
position the tape at the beginning of the last file; i.e.\ in the
beginning of the last backup
\begin{verbatim}
mt eod
mt bsf 1
tob -verbose 
\end{verbatim}

Notice that many of the positioning commands of \cmd{mt} does not move
the tape immediately, the tape is first moved when you execute the read
or write command for the tape (\cmd{tob} or \cmd{afio} or whatever you
use directly).

\end{document}